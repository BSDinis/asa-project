\documentclass[a4paper, 12pt, conference, portuguese]{ieeeconf}      % Use this line for a4 paper

\IEEEoverridecommandlockouts                              % This command is only needed if
\overrideIEEEmargins                                      % Needed to meet printer requirements.

% See the \addtolength command later in the file to balance the column lengths
% on the last page of the document

% The following packages can be found on http:\\www.ctan.org
\usepackage{graphics} % for pdf, bitmapped graphics files
\usepackage{epsfig} % for postscript graphics files
\usepackage{mathptmx} % assumes new font selection scheme installed
\usepackage{times} % assumes new font selection scheme installed
\usepackage{amsmath} % assumes amsmath package installed
\usepackage{amssymb}  % assumes amsmath package installed
\usepackage[left=3cm,right=3cm]{geometry}
\usepackage[utf8]{inputenc}
\usepackage[portuguese]{babel}

\title{{\LARGE \bf Cálculo de Sub-Redes e Pontos de Articulação}
\large{Relatório do 1º Projecto - Análise e Síntese de Algoritmos}
}


\author{Baltasar Dinis, 89416 e Afonso Ribeiro, 86752}


\begin{document}

\maketitle
\thispagestyle{empty}
\pagestyle{empty}


%%%%%%%%%%%%%%%%%%%%%%%%%%%%%%%%%%%%%%%%%%%%%%%%%%%%%%%%%%%%%%%%%%%%%%%%%%%%%%%%
\begin{abstract}
  Uma rede deve ser desenhada de forma a que seja resiliente a falhas de
  componentes individuais, permitindo ao seu utilizador uma utilização contínua
  do sistema. A existência de pontos únicos de falha\footnote{\textit{Single
  Points of Failure} em inglês} compromete esta resiliência, impedindo que a
  rede consiga escalar, face ao tráfego. Neste relatório, expomos uma
  metodologia para automaticamente verificar se uma rede é resiliente e quantas
  sub-redes estão presentes na mesma. Este trabalho foi realizado no contexto da
  Unidade Curricular de Análise e Síntese de Algoritmos, no ano lectivo de
  2018-2019.
\end{abstract}


%%%%%%%%%%%%%%%%%%%%%%%%%%%%%%%%%%%%%%%%%%%%%%%%%%%%%%%%%%%%%%%%%%%%%%%%%%%%%%%%
\section{INTRODUÇÃO}\label{intro}
Uma rede deve ser capaz de tolerar falhas individuais e independentes dos seus
componentes. Só assim é que é possível que um serviço que a utilize escale com o
tráfego. Se a rede tiver pontos que, ao falharem, causarem que uma sub-rede
(conjunto de roteadores tal que é possível encontrar um caminho, e
consequentemente enviar uma mensagem de um para o o outro, entre qualquer par)
se divida em várias então não se pode considerar que a rede tem uma topologia
resiliente. Note-se que, caso dois roteadores, para se ligarem um ao outro,
tiverem de se ligar a um destes pontos únicos de falha, significa que qualquer
caminho que ligue estes dois roteadores inclui necessariamente este ponto único
de falha, aumentando então o volume da carga exercida no mesmo.

É também útil conseguir saber quais são as sub-redes existentes numa dada
topologia de roteadores, e quantas existiriam caso fossem retirados os pontos
únicos de falha.

Neste relatório, apresentaremos a nossa solução para este problema. A estrutura
do relatório é a seguinte. Em II,
fazemos um mapeamento do problema para um problema de grafos e
expomos a nossa solução. Em III, fazemos uma análise teórica do
algoritmo empregue, nomeadamente em termos de complexidade. Em
IV apresentamos os resultados da nossa avaliação experimental e
em IV comentamos a correspondência entre os resultados
experimentais e os valores teóricos.

\section{DESCRIÇÃO DA SOLUÇÃO}\label{sol}

\section{ANÁLISE TEÓRICA}\label{theoric}

\section{AVALIAÇÃO EXPERIMENTAL}\label{experimental}

\section{CONCLUSÃO}\label{conclusion}

\begin{thebibliography}{99}

\bibitem{c1} G. O. Young

\end{thebibliography}
\end{document}
